\section{Discussion}
\label{sec:discussion}

How could this be used?

how to move forward? (sneak preview of google summer of code) - the simplification problem can be seen as a problem of the elimination of banana

incorporate further data (ideas: directionality; street names; angles; land use; ...)
use network formalism: on dual approach (intersections = edges): jiang 2004, yang 2022,
rosvall/sneppen; barthelemy paper on shortest path shape

end with a call to action \& 'towards open urban data science'

\bigskip

include in future work:
\begin{itemize}
    \item analyze other regularities in distribution
    \item !! once banana has been found: how to replace it?
    \item non-parametric bandwidth selection
    \item using data on land use of potential ``bananas'' to identify whether they are urban blocks or not - would be great IF data was there (as discussed by Fan et al.  \cite{fan_polygon-based_2016})
\end{itemize}


\subsection*{Data and code availability}

The fully documented workflow, all input data
and all results are made available in open source format on GitHub:
\href{https://github.com/martinfleis/bananas}{github.com/martinfleis/bananas}.