\section{Introduction}
\label{sec:intro}
% COMMENTED OUT - NOTES:
% - Importance of street networks for urban analysis
%     - talk about availability of data, different use cases from transport to
%       morphology to ...
%     - try to illustrate the wide applicability so we can then base the claims about
%       the importance of the issue on top of it
%     - general motivation - framework of urban data science. why everyone would benefit
%       from having this issue solved. cite arcaute on 'recent advances, lobo on 'urban
%       science'. also: alessandretti 2020, louail 2015, barthelemy books (morphogenesis
%       2018; spatial networks 2022)
% - Data need to look different for transport than for morphology ans why it matters
%     - Networks vs polygons enclosed by networks (blocks? negative space? we need to
%       pin down the terminology we want to use)
% - Problem description
%     - Each network comes with a different detail and generated "blocks" are not always
%       what they seem to be but sometimes are an artifact of transport-focused geometry
%     - cite cardillo, geisberger, morer (computational costs), maybe venerandi 2016?;
%       vanegas paper on actually *simulating* these spaces
% - Examples - other authors complaining about the issue, without having solved it yet
%   (e.g. vybornova 2022; grippa 2018; peponis 2007 merges these into urban blocks
%   (replacing by center lines)
%     - include morphometric literature here - mention `momepy.Blocks` algorithm that
%       attemps to go around the issue in a specific way (but does not solve it)
%         - (fleischmann, porta, dibble, etc.) diet 2018 on planar map classification.
%           sharifi on urban forms.
%     - description/terminology: cf. hermosilla 2014 'UBRSA'; see strano 2012 for power
%       law of "land cells" (spaces surrounded by street segments); most recent: shpuza
%       2011, 2017, 2022 (how to get the PDF...). circular compactness - inspired by
%       louf; see also more recent barthelemy 2017 with the same figures;
% -  summary of what happens in this paper
%     - 'towards an automated detection of bananas'; method inspired by louf and
%       barthelemy; tried out on 150 cities across the world

Cities have been the object of scientific inquiry for thousands of years \cite{vitruvius_vitruvius_1999}, particularly for their growth dynamics, population development, and spatial structure. Within the last 50 years, powered by the emergence of Big Data and computational power, data-driven approaches to the study of cities have gained importance. In this context, the urban street networks have proven to be a particularly useful type of data. Street networks are line-based abstractions of the street space, containing information on the connections between intersections and street segments. They are popular objects of urban analyses due to their explanatory power and simplicity (digitizing a street network is much easier than digitizing buildings). At the same time, feasibility of studies that take street networks as a point of departure has greatly increased with open source GIS data becoming available on platforms like OpenStreetMap \cite{arcaute_recent_2021}. Street networks have a wide range of applications, from transportation network design \cite{farahani_review_2013}, assessment of urban sprawl \cite{barrington-leigh_global_2020} or evolution and distribution of street patterns \cite{boeing_multi-scale_2018,boeing2020off} to the classification of urban form \cite{araldi_street_2019,fleischmann_methodological_2021}. 

Street network data sets vary greatly in their level of detail and data quality [CITE]. Whether a street network data set is fit for purpose depends to a large extent on the specific application. For example, traffic routing applications require adequately represented directionality of edges (street segments) [CITE], while urban morphology studies are based on the shape of urban blocks (i.e. the polygons between the edges), and thus do not require the edges to be directed \cite{dibble_origin_2019}. A frequent street network data processing challenge, common to many applications, is to reduce granularity of detail without loss of relevant information and introduction of new imprecisions. In many cases, deciding which information to keep and which to aggregate is easy for a human, but challenging for an algorithm (see Figure \ref{fig:01}). This is true both for studies that are concerned with the street network itself, and for studies that look at the polygons enclosed by the street network. Further complexity is added by the fact that the level of detail of the input network might vary greatly depending on the data source. 

In short, working with street networks entails not only valuable insights, but also several much-lamented, yet unresolved methodological challenges. In this article, we aim to tackle one of these challenges, namely the detection and removal of what we call \textit{street network face artefacts}, or short: \textit{face artefacts}, as explained in detail in the section below. 

The paper is organized as follows: In the next sections, we expand on the problem and the key questions, followed by a review of literature and terminology. We then introduce our methodology, proposing a cheap computational heuristic which allows the automatized identification of face artefacts based on a shape index. We then briefly describe our workflow for face artefact detection, which we apply to XXX cities across the globe. Lastly, we present our results and conclude by discussing implications and outlining potential further steps.

% SNIPPETS REMOVED FROM THE PARAGRAPHS ABOVE

% When requiring the latter, the conversion of geospatial features into a morphological street network, sometimes referred to as ``network simplification'', still poses several unresolved methodological questions to the research community. 

% In this article, we describe one phenomena arising from the usage of transportation-focused mapping of urban space and its potential simplification, splitting the network-based ``face polygons'' into those representing urban blocks and those that do not.

\subsection*{Problem description}
Each geospatially encoded street network comes with a certain level of granularity and a
focus on specific elements of street space. While all attempt to capture primarily connectivity, the resulting graphs can vastly differ. In graph theory, the polygons enclosed by the network edges in a planar space are called \textit{(graph) faces} [CITE]. A detailed look at these face polygons for a given street network often reveals artifacts of transport-focused geometry, as illustrated in Figure \ref{fig:01}.

\begin{figure}
    \centering
    \includegraphics{figures/todop}
    \caption{[A close-up look at [city, location]. The black lines are network edges (street segments). Grey polygons are correctly identified urban blocks; the red polygon is a face artefact that should not be identified as urban block.]}
    \label{fig:01}
\end{figure}

Unlike its grey-colored neighbors, the face polygon colored in red is not an urban block enclosed by streets; rather, it appears in the network due to the representation of a bidirectional street as two separate edges, one in each direction of traffic flow. This way of network representation is suitable for car-based routing, but much less suitable for other applications. If our goal is to generate polygons that are representative of urban blocks and inversely to ensure that the graph is representing the morphological network rather than the transportation network, we can call such face polygons polygons ``face artifacts'', as they occur only as a result of the data preparation model not suited for the purpose. Face artifacts pose a twofold problem. First, for studies concerned with urban form, they introduce a false signal into the distribution of urban shapes and distort the actual shape of their neighboring polygons. Second, for studies concerned with the properties and pattern of the street network, face artefacts introduce superfluous network edges, thus distorting all network metrics based on node degree and/or shortest path computations. A further aggravating factor is that the extent to which face artifacts distort results depends on the analysis conducted, and cannot be quantified without prior identification of such polygons. Thus, no matter whether one is interested in the urban street network or in urban shapes enclosed by the network, the face artifact should be removed as part of data preprocessing, and replaced by a single network edge. Human manual identification of face artefacts would be unambiguous, but prohibitively costly, not scalable and not entirely reproducible. Although this issue has been pointed out by many authors (see section below), a fully automatized approach to the removal of face artefacts is, to our knowledge, still non-existent. We therefore pose the following research question:

\begin{center}
\textit{How can face artifacts in an urban street network be computationally identified?}
\end{center}

In this article, we propose a method to answer this question and test the proposed method’s universality. 

\subsection*{Literature review}

While face artefacts are a commonly known problem in the research community, there is a lack of coherent terminology for the phenomenon. Previous studies have referred to the same issue in widely varying terms, which makes it substantially more difficult to conduct a comprehensive literature review on the topic. Hereby, we apologize for any involuntary omissions of previous work on face artefacts, and simultaneously wish to contribute to a future homogenization of the terminology.

Few studies that explicitly tackle the ``face artifact'' issue could be identified. 

Li et al. \cite{li_polygon-based_2014} point out the difficulties of extracting multilane roads from OpenStreetMap (OSM) that arise from each lane being represented as a separate linestrings. The authors propose a method to identify and merge so-called ``multilane polygons'', i.e. adjacent polygons covering a single street area that result from mapping of multiple street lanes, through a SVM (support vector machine) machine learning algorithm that uses five shape parameters as input. While this method does succeed in identifying face artefacts at multilane roads, it is only reproducible by users with advanced machine learning skills; furthermore, the method requires input of manually classified training data, which adds a substantial amount of effort.

In their study on feature matching between OSM and reference data, Fan et al. \cite{fan_polygon-based_2016} identify face artifacts, which they call ``non-urban block polygons’’, as a data preprocessing issue. The authors use the SVM approach developed by Li et al. \cite{li_polygon-based_2014}, as mentioned above, to identify face artefacts; they point out that the approach fails for smaller face artefacts at traffic junctions.

Sanzana et al. \cite{sanzana_decomposition_2018} elaborate on the process of deriving hydrological response units from drainage networks and find that error correction is needed for so called ``bad-shaped polygons''. One of the subcategories of bad-shaped polygons, as classified by the authors, is ``sliver polygons'', formed mainly by roads and footpaths. 

Grippa et al. \cite{grippa_mapping_2018} classify polygons derived from OpenStreetMap street network data into ``urban blocks'' and ``sliver polygons'' and present a semi-automated workflow, partially in PostGIS, for sliver polygon removal from the data set. Ludwig et al. \cite{ludwig_mapping_2021} take up this approach within the context of land use classification and additionally filter polygons based on a size threshold. 

Vybornova et al. \cite{vybornova_automated_2022} refer to the network pattern that creates ``face artifact’’ as ``parallel edges’’ and present a network shortest path-based approach for their identification, but no solution to effectively remove these from the network.

Related, but not identical to ``face artifact'' is the problem of formalization of street space. In this regard, Peponis et al. \cite{peponis_measuring_2007} conduct an analysis of urban spatial profiles and point out that their input data includes street center lines, but lacks street widths, hence street surfaces are merged into urban blocks. 

In a methodologically different approach, Hermosilla et al. \cite{hermosilla_using_2014} develop a method to derive so-called urban block related street areas (abbreviated by the authors as ``UBRSA''), defined as the street area surrounding an urban block. This method, however, requires urban block boundaries as an input.

Lastly, a recent study by Shpuza \cite{shpuza_shape_2022} describes elongated urban blocks that are delimited either by a street or another type of obstacle (e.g. a waterbody), and that contain no buildings, as ``edge blocks''. Edge blocks can be identified as outliers in a so-called shape matrix based on two geometrical parameters, relative distance and directional fragmentation. However, edge blocks represent actual urban blocks rather than scattered parts of the street space.

\subsection*{A side note on terminology}

Some recent studies describe the ``face artifact'' phenomenon as sliver polygons \cite{grippa_mapping_2018, sanzana_decomposition_2018, ludwig_mapping_2021}. 

However, ``face artifact'' arise as consequence of a context-dependent redundancy of mapped line features, while sliver polygons stem from mismatching boundaries in vector overlays of polygon features \cite{goodchild_statistical_1978, fischer_using_1993, delafontaine_assessment_2009}. The other suggestions available in literature are not suitable either. ``Multilane polygons'' \cite{li_polygon-based_2014} or ``parallel edges’’ \cite{vybornova_automated_2022} do not reflect other transportation geometries causing the issue (e.g. complex intersections) while ``bad-shaped polygons'' use a relatively vague term ``bad'' that does not indicate the actual issue. In addition, ``face artifact'' are, in line with our problem definition, confined to the specific context of urban street networks. Therefore, in spite of some degree of geometric similarity between the two, we refrain from applying the term ``sliver polygon'' in the ``face polygon’’ context and rather build on more generic terminology derived from graph theory.

% add further thoughts on terminology - depending on what we will choose as term li:
% multilane polygons; fan: road area polygons; shpuza: edge blocks; others: sliver
% polygons...
