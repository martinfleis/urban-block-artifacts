\section{Introduction}
\label{sec:intro}

- Importance of street networks for urban analysis
    - talk about availability of data, different use cases from transport to morphology
    to ...
    - try to illustrate the wide applicability so we can then base the claims about the
    importance of the issue on top of it
    - general motivation - framework of urban data science. why everyone would
    benefit from having this issue solved. cite arcaute on 'recent advances, lobo on 'urban
    science'. also: alessandretti 2020, louail 2015, barthelemy books (morphogenesis 2018;
    spatial networks 2022)
- Data need to look different for transport than for morphology ans why it matters
    - Networks vs polygons enclosed by networks (blocks? negative space? we need to pin down the
    terminology we want to use)
- Problem description
    - Each network comes with a different detail and generated "blocks" are not always
    what they seem to be but sometimes are an artifact of transport-focused geometry
    - cite cardillo, geisberger, morer (computational costs), maybe venerandi 2016?;
    vanegas paper on actually *simulating* these spaces
- Examples - other authors complaining about the issue, without having solved
it yet (e.g. best paper ever \cite{vybornova2022automated}); grippa 2018; peponis 2007
merges these into urban blocks (replacing by center lines)
    - include morphometric literature here - mention `momepy.Blocks` algorithm that
    attemps to go around the issue in a specific way (but does not solve it)
        - (fleischmann, porta, dibble, etc.) diet 2018 on planar map
        classification. sharifi on urban forms.
    - description/terminology: cf. hermosilla 2014 'UBRSA'; see strano 2012 for power
    law of "land cells" (spaces surrounded by street segments); most recent: shpuza 2011,
    2017, 2022 (how to get the PDF...). circular compactness - inspired by louf; see also
    more recent barthelemy 2017 with the same figures;
-  summary of what happens in this paper
    - 'towards an automated detection of
    bananas'; method inspired by louf and barthelemy; tried out on 150 cities across the
    world


\bigskip

INTRODUCTION OF THE INTRODUCTION

Many studies within urban science use the street network of a city as primary input. Recent examples are numerous and cover a wide range of applications: from transportation to urban morphologies [cite]. The feasibility and broad applicability of quantitative urban science studies has greatly increased with open source GIS data becoming available on platforms like OpenStreetMap [cite]. 

However, even when a data set of sufficient quality is available, the conversion of geospatial features into a street network, sometimes referred to as ``network simplification’’, still poses several unresolved methodological questions to the research community. The challenge, in a nutshell, is to reduce granularity of detail without loss of relevant information; in many cases, deciding which information to keep and which to aggregate is easy for a human, but challenging for an algorithm (see Figure 1 [ref]). This is true both for studies that are concerned with the street network itself, and for studies that look at the polygons enclosed by the street network. Further complexity is added by the fact that the requirements towards the input network vary greatly depending on the use case. For example, traffic routing applications require adequately represented directionality of edges (street segments), while urban morphology studies are based on the shape of the polygons between the edges [cite]. 

In this article, we work towards resolving one specific issue within the features-to-network conversion process, which we call ``bananas’’. In the next sections, we describe the problem and define our research question, followed by a literature and terminology review. We then briefly describe the methodology that we apply to 150 cities across the globe, and present on overview of our results. The fully documented workflow, all input data and all results are made available in open source format on github [link]. We conclude with a discussion of implications and potential further steps.



PROBLEM DESCRIPTION



Each geospatially encoded street network comes with a certain level of granularity. A close look at the polygons enclosed by the edges (street segments) of a given network often reveals artifacts of transport-focused geometry. An illustration is found in Figure 1 [ref]. 


[Description of Figure 1: A close-up look at [city, location]. The black lines are network edges (street segments). Grey polygons are ``true urban blocks’’; the red polygon is a ``banana’’.]


Unlike its grey-colored neighbors, the polygon colored in red is not an urban block enclosed by streets; rather, it appears in the network due to the representation of a bidirectional street as two separate edges, one in each direction of traffic flow. The ``banana’’ in Figure 1 [ref] poses a twofold problem. First, for studies concerned with urban form, it introduces a false signal into the distribution of urban shapes and distorts the actual shape of its neighboring polygons. Second, for any study that is concerned with the street pattern rather than with routing, the ``banana’’ introduces a superfluous edge, and it does so in a frequently inconsistent matter – not for all, but only some bidirectional and/or multilane streets. The extent to which the ``banana’’ situation distorts results depends on the analysis conducted, and cannot be quantified without prior ``banana’’ identification. Thus, no matter whether one is interested in the network or in shapes enclosed by the network, the ``banana’’ should simply not be there, but rather be replaced by one network edge. Human manual processing would be unambiguous, but prohibitively costly. We therefore pose the following research question:


RESEARCH QUESTION
How can ``bananas’’ in an urban street network be computationally identified?


LITERATURE REVIEW AND TERMINOLOGY

Disclaimer: We do not claim the literature review to be complete. Rather, we are sure that there are many studies on this which we simply did not find due to the lack of a coherent terminology in the problem description. For this we apologize, and hope to contribute to a future homogenization of the problem discussion.

Chronological review of problem description:

Within the GIS community, a similar problem has been described already in the 1970ies as ``sliver polygon’’. 

For definition see (but dont cite) \url{https://en.wikipedia.org/wiki/Sliver_polygon} 

Goodchild 1978 (in “Statistical aspects of the **polygon overlay** problem) – apparently first time it is described (but can’t get a hold of the article)

Rybaczuk 1993: First description of “information based rules for sliver polygon removal in GIS”

Delafontaine et al. 2009: More recent foundational study on Sliver polygon removal



Sliver polygon removal in QGIS: \url{https://docs.qgis.org/3.22/en/docs/user_manual/processing_algs/qgis/vectorgeometry.html#eliminate-selected-polygons}

Sliver polygon removal in ArcGIS: \url{https://pro.arcgis.com/en/pro-app/latest/help/data/validating-data/polygon-sliver.htm}

TO CHECK: **what criteria** are used to identify sliver polygons? and are they replaced by something upon their removal from the data set? If so, **what criteria** are used to compute that replacing something?

While geometrically similar to sliver polygons, ``bananas’’ stem from a context-dependent redundancy of mapped line features, rather than from mismatching boundaries in overlays of polygon features. In addition, ``bananas’’ are, in line with our problem definition, confined to the specific context of urban street networks; hence our choice of vocabulary to call ``bananas’’ STREET SLIVER POLYGONS.

Studies wrangling with sliver polygons in the context specifically of STREET NETWORKS:

Peponis 2007: conduct an analysis of urban spatial profiles and point out that their input data includes street center lines, but lacks street widths, hence street surfaces are merged into urban blocks. 

Strano et al. 2012: check if there is any reference to this issue? (or is the data they used not fine grained enough to encounter that problem?)

In a related, but methodologically different approach, Hermosilla et al. 2014 develop a method to derive so-called urban block related street areas (abbreviated by the authors as UBRSA), defined as the street area surrounding an urban block. This method, however, requires urban block boundaries as input.

Grippa 2018 et al: classify polygons derived from OpenStreetMap street network data into ``urban blocks’’ and ``sliver polygons’’ and present a semiautomated workflow, partially in PostGIS, for sliver polygon removal from the data set.

Shpuza 2022: [to be read]

Vybornova et al 2022: call them “parallel edges” and present a network path-based approach for identification (but no solution to how to remove these)



